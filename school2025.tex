\title{Madagascar tutorial: Writing technical papers by using Madagascar and LaTeX}

\author{Maurice the Aye-Aye\footnotemark[1]}

\address{
\footnotemark[1] King Julien's Royal Advisor, \\
Island of Madagascar, \\ Africa}

\lefthead{Aye-Aye}
\righthead{Tutorial}
\footer{Madagascar Documentation}

\maketitle

\begin{abstract}
From the viewpoint of science, all technical papers should be
"reproducible" in the sense that someone of reasonable skill ought to
be able to read the paper and then reproduce the results (from Joe
Dellinger). \texttt{SEGTeX} is a \LaTeX\ package for geophysical
publications, which is a important component in \textsc{Madagascar}
Project. \texttt{SEGTeX} consists of class files for Geophysics
papers, SEG expanded abstracts, etc, and cumulative bibliography of
geophysical publications. In this tutorial, you will dive into writing
part in the \textsc{Madagascar} architecture \cite[]{m8r} and learn
how to use the \textsc{Madagascar} and \texttt{SEGTeX} to write a
reproducible paper. By the end of this tutorial, you should have
learned to: 
\begin{enumerate} 
\item design a paper project by following \texttt{SEGTeX} rules,
\item insert figures from \textsc{Madagascar} workflow into papers, 
\item generate a PDF paper with several styles by using different 
\texttt{SEGTeX} classes,
\item revise a paper with notated signs.  
\end{enumerate}
\end{abstract}

\section{Prerequisites}

Completing this tutorial requires
\begin{itemize}
\item \textsc{Madagascar} software environment available from \\
\url{http://www.ahay.org}
\item \LaTeX\ environment with \texttt{SEGTeX} available from \\ 
\url{http://www.ahay.org/wiki/SEGTeX}
\end{itemize}
To do the assignment on your personal computer, you need to install
the required environments. An Internet connection is required for
access to the data repository.

The tutorial itself is available from the \textsc{Madagascar} repository
by running
\begin{verbatim}
git clone https://github.com/cplusv2023/school2025.git  ~/school2025
\end{verbatim}

\section{Introduction}

In this tutorial, you will be asked to run commands from the Unix
shell (identified by \texttt{bash\$}) and to edit files in a text
editor. Different editors are available in a typical Unix environment
(\texttt{vi}, \texttt{emacs}, \texttt{nedit}, etc.)

Your first assignment:
\begin{enumerate}
\item Open a Unix shell.
\item Change directory to the tutorial directory
\begin{verbatim}
bash$ cd ~/school2025
\end{verbatim}
\item Open the \texttt{school2025.tex} file in your favorite editor,
  for example by running
\begin{verbatim}
bash$ emacs school2025.tex & 
\end{verbatim}
\item Look at the first line in the file and change the author name
  from Maurice the Aye-Aye to your name (first things first).
\end{enumerate}

\section{Examples}
\subsection{Basic commands}
\begin{enumerate}
\item Change directory to \texttt{plots} directory
\begin{verbatim}
bash$ cd plots
\end{verbatim}
\item Run
\begin{verbatim}
bash$ scons model.view
\end{verbatim}

in the Unix shell. A number of commands will appear in the shell
followed by Figure~\ref{fig:model} appearing on your screen.
\item To understand the commands, examine the script that generated them by 
opening the \texttt{SConstruct} file in a text editor. Notice that,
instead of Shell commands, the script contains rules.
\begin{itemize}
\item The first rule, \texttt{Fetch}, allows the script to download the 
input data file \texttt{plots.asc} from the data server.
\item Other rules have the form \texttt{Flow(target,source,command)} for
generating data files or \texttt{Plot} and \texttt{Result} for
generating picture files.
\item \texttt{Fetch}, \texttt{Flow}, \texttt{Plot}, and \texttt{Result} 
are defined in \textsc{Madagascar}'s \texttt{rsf.proj} package, which
extends the functionality of \href{http://www.scons.org}{SCons} .
\end{itemize}
\item To better understand how rules translate into commands, run 
\begin{verbatim}
bash$ scons -c model.rsf
\end{verbatim}
The \texttt{-c} flag tells scons to remove the \texttt{model.rsf} file
and all its dependencies.
\item Next, run
\begin{verbatim}
bash$ scons -n model.rsf
\end{verbatim}
The \texttt{-n} flag tells scons not to run the command but simply to
display it on the screen. Identify the lines in
the \texttt{SConstruct} file that generate the output you see on the
screen.
\item Run
\begin{verbatim}
bash$ scons model.rsf
\end{verbatim}
  Examine the file \texttt{model.rsf} both by opening it in a text
  editor and by running
\begin{verbatim}
bash$ sfin model.rsf
\end{verbatim}

  When you view model.rsf in the text editor, you see a history of all
  the programs used to create the file. The \texttt{sfin} program
  lists basic information about the file including data dimensions and
  extents of each axis.
\end{enumerate}

\inputdir{plots}
\plot{model}{width=0.75\textwidth}{Sigmoid model.}


\subsection{Plot modules}
Madagascar has provided various modules for data visualization, including \texttt{sfgraph, sfwiggle, sfgrey, sfgrey3}. The previous run
\begin{verbatim}
bash$ scons model.view
\end{verbatim}
renders a greyscale image with \texttt{sfgrey} and displays it on your screen. You can run the following to try \texttt{sfgraph, sfwiggle, sfgrey3}, respectively:
\begin{verbatim}
bash$ scons sin1.view
bash$ scons modelw.view
bash$ scons ltft.view
\end{verbatim}
Examine the srcipt by opening the \texttt{SConstruct} file and see how this works.

You can also use \texttt{SConstruct} file to configure other command-line tools for your custom programs, such as Python, Matlab, GNU Octave, GMT, etc.. Type and run

\begin{verbatim}
bash$ scons view
\end{verbatim}
to generate these figures one by one.

\subsection{Static figures}
Your can put static figures (\ref{fig:fig1}) in the \texttt{Fig} directory and insert them by \texttt{\textbackslash inputdir\{.}\} and \texttt{\textbackslash plot\{fig1\}\{width=0.75\textbackslash textwidth\}\{Figure name.\}}.

\inputdir{.}
\plot{fig1}{width=0.75\textwidth}{Static figure.}

\section{Revision}
You can use \textbackslash new and \textbackslash old commands to mark the changes in the revised version. For example, the revised text is \new{revised text} and the replaced text is \old{replaced text}. 

\bibliographystyle{seg}
\bibliography{school2025}
